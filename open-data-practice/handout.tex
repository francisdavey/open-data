\textbf{{[}Start with} Handout 1: background{]}

Draw attention to the handout and the fact there is a definition of a
database (if people get stuck). Divide into two teams. Ask them to spend
5 minutes thinking about what the database right (team A) or database
copyright (=''copyright in a database'') might be (team B). {[}5
minutes{]}

Now collect their ideas in a feedback session {[}5 minutes{]}

{[}Handout 2: IP rights in databases{]}

Explain these are the correct answers and summarise briefly the two
rights. {[}5 minutes{]}

{[}Handout 3: Football{]}

Ask the teams to decide which rights apply to the fixtures list (team A)
or Football Live data (team B). Take a short poll at the end of the time
to see what each group thought. {[}10 minutes{]}.

{[}Handout 4: Football -- what the courts decided{]}

Discuss the court decisions with the group in light of their answers.
{[}10 minutes{]}

{[}Handout 5: German poems{]}

The teams have less time (and should need less time) to think about
database right (team A) and database copyright (team B) and whether
there was an infringement. {[}5 minutes{]}

\emph{Note: if time is pressing the last exercise can be given out and
left for discussion after the class or moved to after the break and the
time taken up in the buffer time at the end.}

{[}Handout 6: German poems -- what the courts decided{]}

Feedback {[}5 minutes{]}

{[}Break -- 15 minutes{]}

{[}Handout 7: Virality and Criminality{]}

Give a short presentation to handout 7 on virality of licences {[}5
minutes{]} and access control {[}5 minutes{]}

{[}Handout 8: Summary of open licences{]}

Handout 8 isn't for discussion, just as a useful reference for the next
exercise.

{[}Handout 9: OpenCorporates data{]}

Ask the two teams to decide how they will control access to and licence
open corporates data. {[}10 minutes{]}

Let each team feed back to the rest {[}10 minutes = 5 minutes each{]}.
There are no ``right'' answers to the Opencorporates discussion.

Concluding discussion {[}5 minutes{]}

\emph{There is a 10 minute ``buffer'' at the end of the session to allow
for slippage.}

Handout 1

Background

This session assumes that you already have some background in open data
and there isn't time in this short session to go into legal details.
This sheet contains a few headline legal points.

\begin{itemize}
\item
  Most of this session will relate to the law deriving from the Database
  Directive
\item
  We will be discussing two rights:

  \begin{itemize}
  \item
    database right
  \item
    database copyright (or copyright in a database)
  \end{itemize}
\item
  This applies to databases made on or after \textbf{1 January 1998}
  (for older databases, the law is more complicated)
\end{itemize}

A \textbf{database} is defined as:

\begin{quote}
\emph{a collection of independent works, data or other materials
which---}

\emph{(a) are arranged in a systematic or methodical way, and}

\emph{(b) are individually accessible by electronic or other means.}
\end{quote}

In short: almost anything you would think of as a database is a database
for legal purposes along with other things such as poetry anthologies
and libraries which you might not think of as a ``database''.

\textbf{Contents v database}

All we are going to care about is the database itself. The contents of
the database may be protected in other ways, in particular by copyright
law. The only copyright law we will look at today will be copyright in
the database itself and not the contents of the database.

Handout 2

IP rights in databases

\textbf{Database Right }

\begin{itemize}
\item
  Protects \emph{investment}.
\item
  Requires \emph{substantial investment} in at least one of the
  following:

  \begin{itemize}
  \item
    \textbf{obtaining} the contents
  \item
    \textbf{verifying} the contents
  \item
    \textbf{presenting} the contents
  \end{itemize}
\item
  ``substantial'' could be:

  \begin{itemize}
  \item
    qualitative (eg using highly skilled volunteers)
  \item
    quantitative (eg investing lots of money)
  \end{itemize}
\item
  15 years since database was last updated
\end{itemize}

\textbf{Database Copyright}

\begin{itemize}
\item
  Requires one or more \emph{authors}.
\item
  Protects creative authorship.
\item
  Requires that the author(s) has invested their \emph{own intellectual
  creation} in one of the following:

  \begin{itemize}
  \item
    \textbf{selecting} the contents
  \item
    \textbf{arranging} the contents
  \end{itemize}
\item
  Life of author + 70 years from date database was created
\end{itemize}

Handout 3

Exercise - Football fixtures

Team A: \textbf{Football Fixtures}

The English Premier League has 20 teams, each playing 38 matches each
season, giving a total of 380 matches. The three other leagues have 24
teams and a correspondingly larger number of matches. The Scottish
League is smaller.

The matches have to be organised subject to lots of constraints such as
general rules (eg no 3 consecutive home or away matches); requests from
clubs (eg Chelsea playing away during the Notting Hill Carnival); or
other competitions (such as the FA Cup).

Mr Thompson, employed by the English and Scottish Leagues, spends a
significant amount of time and skill preparing the list each year with
quite a bit of help from a computer.

Do the leagues have database right or database copyright over the list
of fixtures?

Team B: ``\textbf{Football Live''}

\begin{itemize}
\item
  Football analysts attend football games
\item
  Send in reports by mobile phone
\item
  Collect data such as:

  \begin{itemize}
  \item
    goals, times of goals, scorers, fouls, substitutions
  \item
    dominant player for 10 minute periods, man of the match, severity of
    a foul
  \end{itemize}
\item
  Costs £600,000 per season.
\end{itemize}

Do the leagues have database right or database copyright over Football
Live?

Handout 4

Football -- what the courts decided

\textbf{Football Fixtures Lists}

No Database Right -- Fixtures Marketing v Organismos prognostikon agonon
podosfairou C-444/02 (European Court of Justice)

\begin{itemize}
\item
  Did not ``obtain'' the data -- created it
\item
  No substantial independent effort in verifying it
\end{itemize}

No Database Copyright -- Football Dataco v Yahoo! UK C-604/10 (European
Court of Justice)

\begin{itemize}
\item
  Database copyright is about the structure not contents
\item
  Intellectual effort and skill in creating data irrelevant to database
  copyright
\end{itemize}

\textbf{Football Live Data}

Database Right -- Football Dataco v Stan James (Court of Appeal of
England and Wales)

\begin{itemize}
\item
  Facts (such as the scoring of a goal) are pre-existing data which is
  ``obtained'' by the football analysis on behalf of Football Dataco
\item
  Directive clearly intended that database right would protect this kind
  of data, otherwise it would be a ``mouse of a right''
\item
  Unmeasured information (eg a temperature) is an ``existing independent
  material''
\end{itemize}

Database copyright assumed not to exist.

\textbf{Handout 5}

\textbf{Infringement}

Owning database right or database copyright means that your permission
is required before someone else may do certain things with the database.
If they do not have your permission they ``infringe'' your right.

\textbf{Database right}

Database right is infringed (amongst other things) by:

\begin{itemize}
\item
  Extraction (``the permanent or temporary transfer of all or a
  substantial part of the contents of a database to another medium by
  any means or in any form'')
\item
  Re-utilization (``any form of making available to the public all or a
  substantial part of the contents of a database by the distribution of
  copies, by renting, by on-line or other forms of transmission'')
\end{itemize}

But note that it is not infringed by making a report, infographic, map
or diagram based on the data, provided the result does not contain a
substantial part of the original data.

\textbf{Database copyright}

Copyright is infringed by (amongst other things):

\begin{itemize}
\item
  Copying
\item
  Making an adaptation
\end{itemize}

\textbf{Handout 6}

Exercise - German poems

Dataset: the ``1,100 most important poems in German literature between
1730 and 1900''.

A German university (Albrecht-Ludwigs-Universität Freiburg) created this
list of poems in two stages:

\begin{itemize}
\item
  A German academic created a list of titles of German poems that (in
  his view) were important.
\item
  A representative corpus of German-language poetry anthologies was
  selected and all 20,000 poems in those anthologies had their titles
  and first lines standardised.
\item
  A statistical analysis was run to determine how frequently each poem
  was referenced, only those more frequently mentioned were selected
\item
  A publication was created listing the poems in order of frequency of
  mention and giving their title, author, opening line and year of
  publication
\item
  The whole process took 2.5 years and cost 34,900 Euro
\end{itemize}

Directmedia has their own collection of 1000 poems. Directmedia staff
consulted the 1,100 poem list for inspiration and ended up with 856
poems in common.

Team A: Has there been an infringement of database copyright?

Team B: Has there been an infringement of database right?

Hint: make sure you consult handout 5.

Handout 7

German poems -- what the courts decided

\textbf{Database Right}

Yes. The Database right was infringed by \emph{extraction} --
Directmedia Publishing v Albert-Ludwigs-Universitt Freiburg C-304/07
(European Court of Justice)

\begin{itemize}
\item
  Extraction can be done by hand (no computer is needed)
\item
  It doesn't matter that the data is presented in a different form
\end{itemize}

\textbf{Database Copyright}

Yes. Database copyright was also infringed. (German Federal Court of
Justice)

Handout 8

Virality and Criminality

\textbf{Viral Licences}

\emph{A viral licence on a work forces others to openly share works they
create from that work.}

Example: CC-BY-SA (Creative Commons Attribution Share-Alike)

If I write a book and license it under CC-BY-SA, anyone else can freely
copy it and publish it. However, if they publish it they must also
license it under a compatible open licence. If they make a derived work
from it, they must also license that under a compatible open licence --
hence the name ``viral'' because the ``openness'' is catching.

Application: a viral licence allows wide dissemination and re-use of
your work amongst the community, eg of developers, but anyone wanting to
commercially exploit it may want to be able to impose a non-open licence
on their product. To do so they would need to obtain another (possibly
non open) licence from you.

\textbf{Criminality of Access}

\emph{Scraping and other access to a dataset may be criminal in the UK
if it is contrary to the site owner's permission.}

Section 1 of the Computer Misuse Act 1990. It is an offence to access a
computer if:

\begin{itemize}
\item
  the access is unauthorised
\item
  you know the access is unauthorised
\end{itemize}

Example: police officer uses police computer to find out whose car is
parked outside his ex-wife's house.

Application:

If your data is:

\begin{itemize}
\item
  Only useful if it is fresh (eg bus locations)
\item
  Supplied via API using a key or some similar process
\item
  You can control use, by controlling access
\end{itemize}

Handout 9

Summary of open licences

You should already have encountered some well-known open licences. This
sheet summarises features of the main open data licences.

\begin{longtable}[c]{@{}l@{}}
\toprule
\textbf{Unrestricted}\tabularnewline
\midrule
\endhead
\textbf{CC0}\tabularnewline
\textbf{PDDL}\tabularnewline
\textbf{Attribution}\tabularnewline
\textbf{OGL} (Open Government Licence)\tabularnewline
\textbf{CC-BY} (Creative Commons Attribution)\tabularnewline
\textbf{ODC-BY} (Open Data Commons Attribution)\tabularnewline
\textbf{Viral}\tabularnewline
\textbf{CC-BY-SA (from version 4.0)}\tabularnewline
\textbf{ODbL}\tabularnewline
\bottomrule
\end{longtable}

Handout 10

OpenCorporates data

Opencorporates is an open data startup that collects data from numerous
national registries of companies. The data is obtained in different ways
-- often by scraping the source but by no means always. Each registry
makes available different information but it may include:

\begin{itemize}
\item
  Name and registration number of the company
\item
  Name and address of directors
\item
  Documents filed by the company (eg annual reports)
\item
  Accounting information
\end{itemize}

Opencorporates wants to:

\begin{itemize}
\item
  Encourage re-use, particularly for people doing research or making
  other open data websites and services, but also commercially
\item
  Ensure that no-one can simply re-sell their data (and therefore
  compete with them)
\item
  Ensure that significant uses of their computer systems are paid for
\end{itemize}

Discuss how Opencorporates might achieve this by reference to:

\begin{itemize}
\item
  The technical methods they use to publish the data
\item
  The licence(s) used
\end{itemize}
