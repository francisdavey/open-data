\subsection{Introduction}\label{introduction}

This handout is intended to accompany my talk on open data law and
licensing. It does not follow the order of the slides and is intended to
cover more material than it would be possible to discuss in a relatively
short training session.

Some previous attendees of this course have given feedback that the
subtle interplay between court decisions and the law was not something
they were familiar with. I have included a rough guide to jurisdiction
and the creation of law in a section at the end.

In this talk, I will be focussing particuarly on the law of England and
Wales where I practice, but much of what I say will be applicable
throughout Europe and, in some cases, internationally.

\subsection{Ownership of data}\label{ownership-of-data}

Strictly speaking it is impossible to ``own'' data in the sense of
having a property right in factual information.\footnote{Response Ltd v
  Datateam Business Media Ltd
  \href{http://www.bailii.org/ew/cases/EWCA/Civ/2014/281.html}{{[}2014{]}
  EWCA Civ 281}} Nevertheless it is possible to own:

\begin{itemize}
\itemsep1pt\parskip0pt\parsep0pt
\item
  copyright in the \emph{expression} of information (eg in a map or
  diagram)
\item
  database copyright in the selection or arrangement of a collection of
  data
\item
  database right in the investment in obtaining, verifying or presenting
  a database.
\end{itemize}

For this course I will mostly be concerned with rights in the database
rather than rights in other expressions of information, but in
considering how to license a database you should consider how you will
deal with:

\begin{itemize}
\itemsep1pt\parskip0pt\parsep0pt
\item
  the \emph{contents} of the database
\item
  the \emph{database itself}
\end{itemize}

\subsubsection{Copyright}\label{copyright}

\paragraph{Some history}\label{some-history}

Copyright began in Great Britain as a protection for ``books and other
writings''. It restricted the right to copy the books or other writings
(hence the term ``copyright'') to the authors of the books for 14 years
from publication.

Over time copyright protection was extended to many other forms of
expression, for example paintings, sculptures, plays and photographs.
These are collectively known as ``works''. As protection extended it
became unclear what exactly copyright protected. It was generally
understood that copyright shouldn't be able to protect facts. But that
it should be able to protect adaptation of a work, eg translations into
another language, dramatizations of a literary work. But it was not
clear precisely where the line should be drawn

For example, there were some early cases where judges were unsure about
whether maps could be protected since they simply represent what is
already out there. There were also questions over drawings of plants and
(later) photographs. Eventually it was accepted that copyright could
protect all these kinds of work.

The courts converged on a distinction between ``ideas'' -- which were
not protected -- and the ``expression'' of those ideas -- which was
protected. This is known as the ``idea/expression'' dichotomy. You could
take facts recorded in someone's work, provided you did not copy the
expression of the work.

Sometimes this caused difficulties where the work was, say, a table of
information or a compilation. The courts settled on the idea that a
compilation of facts might be protected by copyright even if the facts
themselves were not.

In English law the concept of ``originality'' acted as a kind of
gate-keeper to prevent copyright being too extensive. If a work was not
original, there was no copyright in it.

At first it wasn't clear what kind of originality was required. In
particular was it enough to put a lot of effort into a work where that
work was not, necessarily, particularly creative. For example, creating
a directory of streets in London requires a lot of effort, but it is not
particularly original since anyone could go out and do it. Should
copyright protect that data collection effort?

\subsubsection{European harmonisation}\label{european-harmonisation}

All that argument, though very interesting for academics, is now past.
In Europe, whether or not a work is protected by copyright (the
``subsistence'' of copyright) has been harmonized by three directives:

\begin{itemize}
\item
  the software directive -- for computer programs
\item
  the database directive -- for databases
\item
  the information society directive -- for everything else.
\end{itemize}

In all cases the criterion (replacing the English conception of
``originality'') is that a work is protected if it is its author's ``own
intellectual creation''.\footnote{Some English courts seem to have been
  slow to catch up with this and you will still find decisions talking
  about ``originality''. There are some subtleties here -- particularly
  for photographs and images -- but for our purposes we can treat the
  only criterion that applies as ``own intellectual creation''.} The ECJ
has explained that ``own intellectual creation'' implies a creative
choice made by the author. They must make use of ``formative freedom''
so that the author puts the stamp of their personality on work. If there
is no ``formative freedom'' -- eg if there is only really one way to
express something -- then there can be no copyright.

Even under the new EU criterion a lot of things are capable of being
protected by copyright. For example some (sufficiently creative)
newspaper headlines might be protected.

How copyright protects the structure of databases will be dealt with
next, but open data may involve copyright works as the \textbf{contents}
of a database. In an open data context there is probably a contrast
between:

\begin{itemize}
\item
  pure data -- eg temperature values or GPS co-ordinates (almost
  certainly not protected individually)
\item
  copyrightable works -- eg images, user comments
\end{itemize}

\subsubsection{General features of
copyright}\label{general-features-of-copyright}

Some useful general features of copyright are worth noting:

\begin{itemize}
\item
  no formalities are required -- copyright ``just happens'' when a work
  is created, you do not have to register or claim it in any
  way\footnote{In the US, registration of copyright is optional, if a
    work that has been registered is infringed, the owner may be able to
    obtain a much higher award of damages for the infringement.}
\item
  you do not even have to mark your work with a copyright notice (C) or
  something similar
\item
  copyright belongs to the author in most cases (films and sound
  recordings being an exception)
\item
  \textbf{unless} the author creates the work in the course of
  employment, when it will first belong to the employer
\end{itemize}

\subsubsection{Infringement of
copyright}\label{infringement-of-copyright}

Copyright in a work is infringed by doing any of the following without
permission of the copyright owner:

\begin{itemize}
\item
  reproducing the work (i.e.~copying it)
\item
  distributing the work
\item
  communicating the work to the public (eg publishing it on a web page
  or via an API)
\end{itemize}

\subsubsection{Modern database
protection}\label{modern-database-protection}

For the purpose of the database directive, a ``database'' is

``a collection of independent works, data or other materials which (a)
are arranged in a systematic or methodical way, and (b) are individually
accessible by electronic or other means''

Thus a ``database'' is what a computer scientist might call a ``data
set''. A database does not have to be held on a computer. Examples of
databases which are not electronic might be:

\begin{itemize}
\item
  a library (arranged according to the Dewey Decimal system)
\item
  an anthology of poems organised by author (and date)
\item
  a parish register (arranged by date)
\end{itemize}

If a database is held on a computer, the computer software that manages
the database is not a part of the ``database'' for IP purposes. It would
usually be protected by copyright law as a computer program. Copyright
operates slightly differently for computer programs than it does for
databases and is probably off-topic for this handout.

\subsubsection{Database right}\label{database-right}

I deal with database right first, because it was the first of the two
rights in databases to be dealt with by the European Court of Justice.

Database right is quite unlike copyright. It aims not to reward
authorship, but to reward \emph{investment}. It was created by the
European Union in a, possibly mistaken, hope that it would encourage
investment in the creation of a database infrastructure. The database
right is sometimes referred to in EU circles as the ``sui generis
right'' because it is not like any other kind of right.

The database right is also unknown in the US and generally
internationally. This means that a database protected by database right
in the EU will not enjoy that protection in the US. This has important
implications for European institutions publishing databases to the
internet. It means that a US based institution might be able to
appropriate the database and use it in the US without worrying about
database right.

The database right applies to a database where there has been a
substantial investment in one or more of three factors:

\begin{itemize}
\item
  obtaining
\item
  verifying
\item
  presenting
\end{itemize}

the contents of the database. Here ``substantial'' can mean either
quantitatively substantial -- for example lots of time or money; or
qualitatively substantial -- for example the use of experts to collect
the data (as happened in \emph{Football Dataco v Stan James below}).

\paragraph[Fixtures Marketing v Organismos prognostikon agonon
podosfairou]{Fixtures Marketing v Organismos prognostikon agonon
podosfairou\footnote{C-444/02 (European Court of Justice)}}\label{fixtures-marketing-v-organismos-prognostikon-agonon-podosfairou8}

The Court of Justice explained that ``obtaining'' was different from
``creating''. The football leagues of England and Scotland claimed they
had a database right in the list of football ``fixtures'', i.e.~which
teams would play each other on particular dates and in particular
venues. The court decided that what the leagues were doing was
``creating'' data. It was they that decided whether or not two teams
would play one another. They could not be said to be ``obtaining'' it.

This conclusion surprised quite a few commentators. Databases like the
football league's fixtures list may even have been in lawmakers' minds
when the database directive was drafted. The court of justice explained
that the purpose of the database right was to encourage the investment
in creating database infrastructure. The football league needed no
encouragement via the database right to invest in their database of
fixtures lists. They would have created such a thing anyway.

\paragraph[Football Dataco v Stan James]{Football Dataco v Stan
James\footnote{{[}2013{]} EWCA Civ 27, followed in \emph{Calor v Flogas}
  {[}2013{]} EWHC 3060 (Ch) where even the parts of Calor's customer
  database that were created anew by Calor were thought to be protected}}\label{football-dataco-v-stan-james9}

Where is the line drawn between ``obtaining'' and ``creating''? Some
academics have suggested that a common form of data collection -- going
out and measuring data (eg meteorological data, mapping data for Open
Street Map) -- was really ``creating'' the data on the theory that the
data doesn't exist before it is actually recorded. The Court of Appeal
of England and Wales has rejected this view in no uncertain terms. It is
not certain that the Court of Justice would take the same view, but the
question hasn't been asked them. For the moment a substantial investment
in measuring data would give the database right in English law.

\paragraph{Apis-Hristovich v Lakorda
C-545/07}\label{apis-hristovich-v-lakorda-c-54507}

This was a case about databases of public legal information. For an open
data perspective, the most interesting part of the decision was that
simply because information is collected form publicly accessible data
sources does not preclude it from being protected by the database
right.\footnote{Indeed publicly created data might even be subject to
  the database right of the public body that is curating it, see
  http://www.techdirt.com/articles/20130211/08050521945/europes-database-right-could-throttle-open-data-moves-there.shtml}

The court (re)-iterated a few other points that are worth remembering:

\begin{itemize}
\item
  ``Extraction'' (see below) relies on a transfer of the contents of the
  database. The purpose of that transfer is irrelevant.
\item
  It may be that the contents of database B resemble those of an earlier
  database A. That may be because of a transfer of the contents of A to
  B but there may be other good explanations. There is no rule B
  infringes A just because it looks the same.
\end{itemize}

\subsubsection{Infringement}\label{infringement}

The database right protects the owner against three things being done
without their permission. The first two are:

\begin{itemize}
\item
  extraction; or
\item
  re-utilisation
\end{itemize}

of a substantial part of the database.

But the authors of the directive were also clearly worried about the
possibility of taking ``little and often''. The directive also makes
conduct an infringement of the database right if it consists of all of
the following:

\begin{itemize}
\item
  repeated and systematic;
\item
  extraction and/or re-­utilisation of \textbf{insubstantial} parts of
  the contents of the database;
\item
  implying acts which:

  \begin{itemize}
  \itemsep1pt\parskip0pt\parsep0pt
  \item
    conflict with a normal exploitation of that database
  \item
    or which unreasonably prejudice the legitimate interests of the
    maker\footnote{Article 7(5)}
  \end{itemize}
\end{itemize}

An \textbf{extraction} is involved when the database appears in a new
medium. It does not matter if there has been human intervention -- even
selective intervention -- in between.

For example in the Directmedia Case,\footnote{Directmedia Publishing v
  Albert-Ludwigs-Universität Freiburg C-304/07} a University owned a
database of the ``100 most important poems in German literature between
1730 and 1900. Directmedia created their own anthology, using the
database as a guide to which poems to select from that period. They were
selective and did not use all the poems. The Court of Justice decided
that could be an ``extraction''.

\textbf{Re-utilisation} means making (a substantial part of) the
database available for members of the public to access.

For example in the Innoweb Case,\footnote{Innoweb v Wegener ICT Media BV
  C-202/12} a meta search engine (``GasPedaal'') would, in response to a
user query, make its own queries to a number of automotive websites and
in particular one known as Autotrack. A member of the public could, in
principle, access a substantial part of Auotrack's database by using
GasPedaal and hence the meta search engine's activities constituted
``re-utilisation''.

\subsubsection{Derived works}\label{derived-works}

What the database right does not seem to protect is a non-database work
that has been created using a database -- for example a graphic (such as
a map or info-graphic) or a report summarising the results in a
database. This is a subtle point because, if the derived work contains
sufficient information that was in the original database to count as a
``substantial part'', then creating the derived work will be an
``extraction'', which would require permission of the database right
owner.

But in many cases the derived work will not contain a copy of a
substantial part of the information in the database and so there will be
no extraction or re-utilization of the database.

This means that the degree to which a database right owner can use the
database right to control the \emph{use} to which their data is
subsequently put is limited.

\subsubsection{Copyright in databases}\label{copyright-in-databases}

Copyright in a database is a right independent of any database right. In
other words it is possible for either copyright or database right to
exist in a database as well as both or neither. Neither right is
``superior'' to the other.

To obtain copyright a database must be original, in that one of two
things must be the author's ``own intellectual creation''. Either the:

\begin{itemize}
\item
  selection or
\item
  arrangement.
\end{itemize}

For example the 100 most important poems in German literature between
1730 and 1900 could claim copyright on the basis of the selection of the
poems to include. This would apply to many anthologies. Wherever what is
and is not in the database is the result of the exercise of the author's
``own intellectual creation''.

\subsubsection{Playlists and CD's}\label{playlists-and-cds}

It has been widely reported that the Ministry of Sound have sued the
music streaming company Spotify for infringement of copyright in their
playlists. Oddly, the database directive specifically mentions this sort
of situation in recital 19 which says:

``Whereas, as a rule, the compilation of several recordings of musical
performances on a CD does not come within the scope of this Directive,
both because, as a compilation, it does not meet the conditions for
copyright protection and because it does not represent a substantial
enough investment to be eligible under the sui generis right''

In other words: no database right or database copyright for you. The
phrase ``as a rule'' means that there may be some exceptions, but the
drafters of the directive clearly though that a typical CD could not
normally attract either database right or database copyright.

\subsubsection{Other rights}\label{other-rights}

There are a few other rights that might protect information alone, which
I will deal with briefly here.

\begin{itemize}
\item
  Confidential information -- if information is given in confidence then
  it is a \emph{breach of confidence} to communicate it to someone else.
  Information that has been published will almost never be confidential.
  The law of confidential information is fairly common-sense. For open
  data the take-home is to make sure that data you publish hasn't been
  obtained in breach of confidence or been the subject of confidence at
  some point.
\item
  ``Hot news''\footnote{International News Service v. Associated Press,
    248 U.S. 215 (1918)} -- this is an idea that had some currency in
  the US for a while, though it is probably now much less
  important\footnote{EFF: The ``Hot News'' Doctrine After Fly On the
    Wall: Surviving, But On Life Support (at
    https://www.eff.org/deeplinks/2011/06/hot-news-doctrine-surviving-life-support)}
  -- it was a doctrine that protected ``hot'' (i.e.~time critical) news.
\item
  German Leistungsschutzrecht für Presseverlege (aka ``Lex Google'') a
  right over news content for one year.
\end{itemize}

\subsubsection{Effective technological
measures}\label{effective-technological-measures}

Database may also be protected by various forms of digital rights
management, passwords or other forms of technology intended to control
their use. These are known as ``effective technical measures'' in EU
law. In many situations, circumventing an effective technological
measure can be treated as the equivalent of an infringement of copyright
and in some cases is a criminal offence.

It should be unusual to meet ETM's in the context of open data.

\subsection{Open licensing}\label{open-licensing}

\subsubsection{What is a licence?}\label{what-is-a-licence}

Note that in UK English ``licence'' is a noun and ``license'' a verb. In
US English both are ``license''. The US usage is very common online and
some licences thus have names that look odd to an English lawyer's eye,
eg the Creative Commons License.

A licence is just a permission to someone (``the licensee'') to do
something for which they would need the permission of someone else
(``the licensor''). It need not -- and often is not -- be a contract.

For example: the postal worker who delivers our post walks up our
driveway. If they did not have our permission to do so they would be a
trespasser. The law implies a licence for lawful visitors to go to the
front door of a house or building. This is clearly not a contract and
could be revoked by telling a visitor to go away. The licence would not
be implied where there was a clear notice forbidding entrance (eg ``no
hawkers'').

In the context of copyright, a copyright licence gives permission to do
one or more of the acts restricted by copyright (eg reproduction or
communication to the public). A licence may have conditions attached.
Breach of the conditions would withdraw the permission given by the
licence. In many cases this would result in the licensee in breach of
conditions of the licence infringing copyright.

Licences may also be restricted in various ways, eg they may be

\begin{itemize}
\item
  exclusive or non-exclusive
\item
  limited by jurisdiction (eg a UK licence)
\item
  limited by time
\item
  revocable in certain circumstances (or simply at the choice of the
  licensor)
\item
  subject to the making of regular payments or payments calculated by
  some use made by the licensee (eg royalties)
\end{itemize}

Hence a common bit of boilerplate one sees is:

``a non-exclusive irrevocable, perpetual, royalty-free
worldwide\footnote{We are beginning to see (rightly in my view) the
  adoption of a practice long existing in Hollywood which is to spell
  out that the licence extends beyond planet Earth.} licence \ldots{}''

\subsubsection{What is ``open''?}\label{what-is-open}

The word ``open'' is used (or mis-used) by different groups to mean
different things. For the ODI and for the purposes of today's talk,
``open'' follows the Open Definition:

\begin{quote}
``A piece of data or content is open if anyone is free to use, reuse,
and redistribute it --- subject only, at most, to the requirement to
attribute and/or share-alike.''\footnote{See
  \url{http://opendefinition.org/} The Open Definition is curated by the
  Open Knowledge Foundation (OKF). I declare an interest in that I
  frequently act for the OKF.}
\end{quote}

\subsubsection{Share-alike / viral
licensing}\label{share-alike-viral-licensing}

One of the pioneers of open licensing was Richard Stallman, the founder
of the GNU project.\footnote{GNU is a recursive acronym for Gnu's Not
  Unix. This is a computer science joke.} For Stallman it was important
that users of software had access to the source code so they could make
their own modifications and alterations to it. Those changes might then
be used to create better software for everyone, not just the person who
made the modifications. The problem was if he simply released the source
code into the public domain, then anyone making modifications could keep
any changes to themselves and not contribute back to the general
community.

His solution was to create a licence that permitted others to use his
software, along with the source code, but on condition that if they
passed the software on to others -- whether they sold it or not was
irrelevant -- they had to do make the source code available and do so
under the same licence. The licence would then spread ``virally'' to all
adaptations of the original.

Stallman used the term ``free software'' for his product. Not that it
was free to buy, but that it was free for others to use a re-use. ``Free
as in speech, not as in beer'' is how he put it.

\paragraph{Examples:}\label{examples}

\begin{itemize}
\item
  GPL (GNU Public Licence) -- open source software
\item
  CC-BY-SA (Creative Commons Attribution Share-Alike) -- for works of
  art, literature etc
\item
  ODbL (Open Database License) -- for collections of data
\end{itemize}

There are many non-viral licences. For example:

\begin{itemize}
\item
  CC0
\item
  CC-BY
\item
  PPDL
\item
  Open Government Licence
\end{itemize}

\subsubsection{Multiple licences}\label{multiple-licences}

Where a work is subject to more than one licence any user of that work
will have to consider both licences and try to comply with any
conditions they impose.

For share-alike restrictions, the rule is usually that the same or a
``compatible'' licence is used on any derived work. If a number of works
-- or in the case of data, data sources -- are mixed together then there
is a danger of \textbf{licence incompatibility}. This could make it
difficult or impossible to mix together incompatibly licensed material.

A useful tool (which I haven't checked for legal accuracy) for
discovering whether it is possible to mix licences can be found here at
\url{http://clipol.org/tools/compatibility}. The ODI also makes
available information about licence compatibility at
\url{https://github.com/theodi/open-data-licensing/blob/master/guides/licence-compatibility.md}

For that reason, writing your own open licence from scratch is a bad
idea. It is unlikely to be compatible with other open data licences and
will almost certainly create more problems than it solves.

Rule 1 of open licensing\\ \textbf{Never write your own open licence.}

\subsection{Which licence?}\label{which-licence}

Remember that items of data may be protected by copyright. The whole
collection of data may be protected by database copyright and database
right. You may want to use a different licence for the contents of your
database than for the database as a whole. You might even have to do
that if the contents were licensed to you under specific open licences.

Government -- OGL (now in version 2.0\footnote{See
  https://www.nationalarchives.gov.uk/doc/open-government-licence/version/2/})

\subsubsection{Pure public domain}\label{pure-public-domain}

Both CC0 and the PDDL are strong statements that attempt to give up all
rights in the material licenced. Both cover the database right as well
as database copyright.

\subsection{Creative Commons}\label{creative-commons}

Creative commons offers a number of options

\begin{itemize}
\item
  use restrictions: none, or non-commercial (NC)
\item
  re-use restrictions: none, share-alike (SA) or no derivatives (ND)
\end{itemize}

All options (other than CC0 above) require attribution (BY). Which
results in 6 possible variants:

\begin{itemize}
\item
  BY
\item
  BY-ND
\item
  BY-SA
\item
  BY-NC
\item
  BY-NC-ND
\item
  BY-NC-SA
\end{itemize}

Both NC and ND are \textbf{not} open licence choices since they impose
further restrictions on use. NC is particularly unhelpful because it is
so unclear what counts as ``commercial'' use. Eg, does it apply to a
website which is monetized by advertising? Thus only CC-BY and CC-BY-SA
are ``open'' licences.

Creative Commons have recently introduced version 4.0 of their licence
suite,\footnote{http://creativecommons.org/} which now covers the
database right (and analogous rights in other jurisdictions) as well as
copyright. A CC licence may be applied to any licensed work -- so it may
be applied simply to the structure of a database and not its contents
(in much the same way as ODC licences) -- but unless otherwise noted a
CC licence applied to a database is likely to apply to both the contents
and the (structure, collection etc of the) database.

\subsubsection[Open Data Commons]{Open Data Commons\footnote{Part of the
  OKF}}\label{open-data-commons21}

As well as the PDDL the ODC offers two database licences:

\begin{itemize}
\item
  ODC-By (attribution)
\item
  ODC-ODbL (attribution and share-alike)
\end{itemize}

These apply only to the database and not its contents. If the contents
may contain copyrightable material, you should consider an open licence
for the contents as well -- eg CC-BY or CC-BY-SA.

This means that both ODC-ODbL and CC-BY-SA v4.0 may be used for the
share-alike licensing of a database.

\subsubsection{Creative Commons v Open Data
Commons?}\label{creative-commons-v-open-data-commons}

CC and ODC's share-alike licences (CC-BY-SA and ODC-ODbL) do not operate
in precisely the same way.

In principle, both try to impose a duty to share-alike (i.e.~share under
a suitable licence) any work derived from the licensed work, so that
licensee are forced to share back into the commons the fruits of using
the licensed work. But the two licences do so differently.

ODC-ODbL, contemplates three different kinds of derived work:
``derivative databases'', ``collective databases'' and ``produced
works'':

\begin{itemize}
\item
  a ``derivative database'' is just a new database based on the licensed
  database (eg by modifying, translating or otherwise adapting it)
\item
  a ``collective database'' is a larger database containing the licensed
  database in unmodified form but combined with other independent
  databases
\item
  a ``produced work'' is some other work (eg a report or info-graphic)
  derived from the licensed database (or a derivative database or
  collective database containing the licensed database).
\end{itemize}

Publishing a ``produced work'' requires identification of the original
licensed database~ (as part of attribution). If the produced work was
produced from a derivative database requires that a copy of the
derivative database (or a practical machine-readable copy of the
differences) is published.

By contrast, CC relies on a concept it calls ``adapted material''.
Adapted material is restricted to material that would have required the
licensor's permission to create from the CC-licensed work. As already
discussed, the database right does not restrict the production of
non-database derived works (such as info-graphics) and so such things
would not be ``adapted material''. This means that no share-alike or
attribution obligations attach to that material.

In other words, ODC-ODBL takes a more aggressive approach to ``share
alike'' than CC-BY-SA for non-database works derived from databases.

\subsection{Applying a licence}\label{applying-a-licence}

Applying a licence should be relatively straightforward. From an
intellectual property point of view, you do not need to prove that
someone using your material has agreed to a licence. Rather if they are
using it, they would have to show that they had a licence.

All that is needed is:

\begin{itemize}
\item
  reference to the licensing information in a clear place (eg the footer
  of a website)
\item
  a clear statement of what is licensed and what (if anything) is not
\end{itemize}

Both CC and ODC licences have clear instructions on how to apply their
licences.

\subsection{Non-proprietary
protection}\label{non-proprietary-protection}

Even where data is not ``owned'' or capable of ownership, the proprietor
of a site making the data available may be able to make it difficult for
others to use the data ways undesirable to the proprietor.

\subsubsection{Computer Misuse}\label{computer-misuse}

In the UK it is an offence\footnote{\href{http://www.legislation.gov.uk/ukpga/1990/18/section/1}{Section
  1},
  \href{http://www.legislation.gov.uk/ukpga/1990/18/contents}{Computer
  Misuse Act 1990}.} to access any computer in order to obtain access to
data where:

\begin{itemize}
\item
  access to the data is unauthorised
\item
  the offender knows that it is unauthorised
\end{itemize}

This is particularly relevant to those involved in ``scraping'' data --
that is automatically simulating the action of a web browser and
extracting data from the web pages read. This might, in appropriate
circumstances, extend to accessing a website intending to use the data
there in a way not authorised by the website owner.

The offence is moderately serious -- leading to a maximum prison
sentence of 2 years. It may also be committed at least partially across
jurisdictional boundaries. Provided some part of the offence happens
within the UK (eg if either the computer accessed, or the person
attempting the access is UK based) then there is an offence within the
UK.

This has implications if you use data from an online data source without
the permission of the data provider. It may also be a problem if data
you are supplied withhas been obtained in that way. As a practical
matter, if you run a business relying on data provided by someone else
you should strongly consider negotiating licence terms with them so you
have some confidence their data source will continue to exist or be
provided in a particular way.

The take away lesson for licensing of open data on a website or
otherwise on the internet is that:

\begin{itemize}
\item
  your licence should be clear about what kind of access is allowed
\item
  you can, in principle, prevent particular people using your data by
  expressly prohibiting it in your licence and making it clear that you
  consider any breach of those terms a criminal act.
\end{itemize}

\subsubsection{Other use restrictions}\label{other-use-restrictions}

The most significant other situation in which a restriction on use may
affect the lawfulness of open data is where doing so is in breach of
contract. Eg where someone signs up to terms and conditions on a website
and then downloads data to use in a way forbidden by those terms and
conditions.

That would be a breach of contract for which the website owner could
sue. They might also be able to obtain an injunction (that is, a court
order) to prevent the user from breaching the conditions of the
contract.

\subsection{Liability for data}\label{liability-for-data}

Some forms of open data may create liability for the publisher. The
subject of liability for online content is too broad to be covered in
this course. It is possible, for example, for an open data set to
contain information that amounts to a defamatory statement and thus risk
defamation liability, itself a highly complex subject. Situations of
that kind ought to be rare.

\subsubsection{Personal Data protection}\label{personal-data-protection}

Open data may also be personal data. Personal data protection law is
mostly harmonised across the EU by the Data Protection
Directive.\footnote{\href{http://eur-lex.europa.eu/LexUriServ/LexUriServ.do?uri=CELEX:31995L0046:EN:HTML}{Directive
  95/46/EC}} In the UK personal data is governed by the Data Protection
Act 1998. In the directive, ``personal data'' is defined as:

``any information relating to an identified or identifiable natural
person''

``an identifiable person is one who can be identified, directly or
indirectly, in particular by reference to an identification number or to
one or more factors specific to his physical, physiological, mental,
economic, cultural or social identity''

This means the following are likely to be personal data:

\begin{itemize}
\item
  email addresses
\item
  patient data with the patient's name removed
\item
  chatty blog posts saying things like ``I met my friend Pete today''
\end{itemize}

Personal data law is too complex to deal with in a morning's talk. There
is considerable guidance on the Information Commissioner's website
(possibly too much) and elsewhere. The take-away for open data is that:

\begin{itemize}
\item
  processing personal data requires the satisfying of numerous legal
  requirements contained in the Data Protection Act 1998
\item
  many of these can be avoided with the \textbf{informed} consent of the
  data subject
\item
  suitably anonymising data -- which may also require aggregation --~
  will prevent it being personal data
\end{itemize}

Removing personal identifiers is not usually enough, because the natural
person may still be identifiable. For example removing a patient's name
from patient records but leaving an identification number that could be
used to identify them would not prevent the data from being personal
data. In practice anonymisation will usually require some form of
aggregation of information or some other way to remove any practical
possibility of ``re-identifying'' the data subject.

The Information Commissioner publishes a code of practice on
anonymisation.\footnote{http://ico.org.uk/for\_organisations/data\_protection/topic\_guides/\textasciitilde{}/media/documents/library/Data\_Protection/Practical\_application/anonymisation-codev2.pdf}
An Open Data Institute sponsored initiative the ``UK Anonymisation
Network''\footnote{http://www.ukanon.net/} can give advice on
anonymisation.

Take away for licensing: in most cases ensure that your data does not
contain personal data or that it is properly anonymised.

\subsubsection{Accuracy of data}\label{accuracy-of-data}

Delegates to this course have expressed concern about their liability
for publishing inaccurate data. If the data is supplied under a contract
-- for example where there is a paid-for API -- then there may be a risk
of liability if the contract does not properly make clear the problems
that there may be with data quality. A clearly worded statement together
with a disclaimer of liability should be sufficient.

It is also possible to be liable for inaccuracies in information under
the common law tort of ``negligent misstatement''. If an inaccurate
statement is made such that:

\begin{itemize}
\item
  someone -- the recipient -- relies on that statement for some
  particular purpose
\item
  it was reasonable for them to rely on the statement for that purpose
\item
  the maker of the statement intended the recipient to rely on the
  statement in that way
\item
  the recipient suffers economic loss as a result of their reliance on
  the statement
\end{itemize}

Then the maker of the statement \textbf{may} be liable for the
recipient's loss. The list of criteria is fairly tough. The courts will
generally not think that it is fair to impose a liability unless the
relationship of the parties is fairly close. In most of the reported
cases the maker of the statement is found not to be liable.

For example, although auditors of a company's accounts may owe limited
duties of care to the shareholders, they do not owe that duty to
third-party investors who may be misled into investing as a result of an
inaccurate statement.\footnote{Caparo Industries v Dickman {[}1990{]} 2
  AC 605}

My view is that this is an unlikely form of liability to occur, but it
may be avoided by attacking the ``reasonable reliance'' criterion. When
publishing data that may contain inaccuracies (the almost invariable
rule) a warning that the data may contain inaccuracies etc in much the
same way as a disclaimer in a contract described above, should make it
very unlikely for liability to be found.

\subsection{Jurisdiction and sources of
law}\label{jurisdiction-and-sources-of-law}

Some So this section offers some very brief notes on the point. Readers
who are already familiar with the law in general (including lawyers)
should jump to the next section.

\subsubsection{Jurisdiction}\label{jurisdiction}

The legal world is, broadly speaking, divided into ``jurisdictions''. In
each jurisdiction the law is different and interpreted by different sets
of courts. In the UK the jurisdictions are:

\begin{itemize}
\item
  England and Wales
\item
  Scotland
\item
  Northern Ireland
\end{itemize}

Scots law is much less like the law of the rest of the UK, but it still
has a great deal in common with English law.

Some legislation in the UK is UK-wide: for example copyright and
immigration law. Other legislation is specific to only parts of the UK
(eg Scotland and England and Wales share the same employment law).

Some countries contain a single jurisdiction -- eg France. In other
countries quite the reverse is true. The UK, with 3, is an example.
Another is the United States where there are 51 different jurisdictions
(one per state plus the District of Columbia).

In the UK there is also an overlapping jurisdiction, that of the
European Union. Where EU law and UK law conflict, EU law takes
precedence. This has similarities with Federal law in the US which is
overlaps (and overrides) state law.

\subsubsection{Sources of Law}\label{sources-of-law}

Law consists mostly of a combination of two things.

First written statements of the law by legislators. In the UK these
include Acts of Parliament (also known as ``statutes'') as well as a
motley collection of what is known as ``secondary'' legislation with
titles such as regulations, rules, orders and so on. Most (but not all)
forms of secondary legislation are made in a standard form known as a
``statutory instrument'' which requires unique numbering and
publication.

In Scotland the Scottish Parliament can also make its own ``Acts of the
Scottish Parliament''.

The other significant component of law is the interpretation of the law
by the courts.\footnote{In Scots law there are some significant academic
  works by the so-called ``institutional writers'' which are treated as
  being formal statements of law.} In the UK a great deal of law was
once made entirely by the courts working from various first principles
and without any legislation to go on. For example the law of breach of
confidence was entirely made up by the courts of England and Wales
without help from any legislators. On the other hand, copyright and
database rights are entirely based in legislation, though they have been
subsequently interpreted by the courts in some cases in great detail.

A substantial amount of the law relating to databases is part of
European law. For our purposes, European law (the law of the European
Union) consists primarily of three forms of legislation: the treaties
setting up the EU, regulations and directives. In theory regulations are
laws that apply directly in member states and directives are laws that
need to be ``transposed'' into the law of a member state. The idea being
that a directive will set out a template for the law, but some detail
will be left to the legislators in each member state to fill in.

For example, the Database Directive\footnote{\href{http://eur-lex.europa.eu/smartapi/cgi/sga_doc?smartapi!celexapi!prod!CELEXnumdoc\&numdoc=31996L0009\&model=guichett\&lg=en}{Directive
  96/9/EC}} was transposed into UK law by the Copyright and Rights in
Databases Regulations 1997.\footnote{See:
  http://www.legislation.gov.uk/uksi/1997/3032/contents/made}

\subsubsection{Precedent}\label{precedent}

In the UK, courts adopt a system known as ``precedent''. This arranges
the courts into a hierarchy. If a court higher up the hierarchy makes a
decision about a point of law, courts lower in the hierarchy are bound
by that decision and cannot depart from it. Courts at the same level
will usually follow earlier decisions of courts at their level (though
the law on whether they have to do so is extremely complex and not easy
to describe in a short series of notes).

In England and Wales the hierarchy is:

\begin{enumerate}
\def\labelenumi{\arabic{enumi}.}
\item
  The Supreme Court (formerly the House of Lords)
\item
  The Court of Appeal
\item
  The High Court (and some superior tribunals)
\end{enumerate}

Below the high court are various other courts (magistrates' courts and
county courts) which are not sufficiently important to set precedents at
all.

Courts will still take seriously decisions made by other courts which
they respect.

\subsubsection{The European Court of
Justice}\label{the-european-court-of-justice}

The European Court of Justice (ECJ or sometimes now CJEU for Court of
Justice of the European Union) has an unusual role. Although it can and
does adjudicate disputes between member states and individuals with the
EU, but its most significant role is in answering questions referred to
it by courts in the Member States.

For example, in the \emph{Fixtures Marketing} case (see below), a court
in Athens asked the ECJ three questions about the database right. The
ECJ answered them and left the final decision on the case to the court
in Athens.

The ECJ is often asked to decide questions that depend on the particular
facts of the case and it routinely makes clear that it is up to the
national court to make those decisions. All the ECJ does is to clarify
points of EU law that are referred to it.

The ECJ's decisions on EU law bind all courts in the Member States. It
does not consider itself bound by its own previous decisions but it
would be unusual for it to depart from a clear line of case law it had
established.
