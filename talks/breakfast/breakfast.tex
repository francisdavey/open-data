\documentclass{beamer}
\usetheme{FD}
\usepackage{ulem}

\newenvironment{blockquote}{%
  \par%
  \medskip
  \leftskip=4em\rightskip=2em%
  \noindent\ignorespaces}{%
  \par\medskip}


\title{Understanding law and licensing\\
collecting, using and publishing data}
\author{Francis Davey}

\begin{document}

\begin{frame}
  \titlepage
\end{frame}

\begin{frame}{Overview}
Who is this directed at?
  \begin{itemize}
  \item Consumers
    \begin{itemize}
    \item Collecting
    \item Using
    \end{itemize}
  \item Producers
    \begin{itemize}
    \item Publishing
    \end{itemize}
  \end{itemize}
\end{frame}

\begin{frame}{Topics}
Only two topics today.
  \begin{itemize}
  \item Property rights in data
  \item Access rights to data
  \end{itemize}
\end{frame}

\begin{frame}{Property Rights}
  \begin{itemize}
  \item Rights over the data (nearest thing to ``ownership'')
    \begin{itemize}
    \item database right
    \item database copyright
    \end{itemize}
  \item Rights of access to data (eg API's)
  \item {\bf NOT} other forms of intellectual property rights
    \begin{itemize}
    \item eg copyright in images, user contributed text
    \item too broad a topic
    \end{itemize}
  \item European Union only
    \begin{itemize}
    \item EU more protective that US in general
    \end{itemize}
  \end{itemize}
\end{frame}

\begin{frame}{IP in Pure Data}
  Database Directive (96/9/EC)
  \begin{blockquote}
    a collection of independent works, data or other materials which (a)
are arranged in a systematic or methodical way, and (b) are individually
accessible by electronic or other means
  \end{blockquote}
  \begin{itemize}
  \item Lots of things are databases (libraries, poetry books,...)
  \item Distinguish the database from its contents
  \end{itemize}
\end{frame}

\begin{frame}{Database Right}
  \begin{itemize}
  \item {\bf Substantial investment}
  \item in:
    \begin{itemize}
    \item obtaining
    \item verifying
    \item presenting
    \end{itemize}
  \item Rewards investment
  \item ``Substantial'' can be qualitative as well as quantitative
  \end{itemize}
  
\end{frame}

\begin{frame}{Football}
The Football leagues have much data.
  \begin{itemize}
  \item Fixtures lists \\
    {\it  Arsenal v Man City 12:45 (13 September 2014) }

\item Football data
  \begin{itemize}
  \item when goals were scored
  \item who was ``man of the match''
  \item ...
  \end{itemize}
  \end{itemize}
\end{frame}

\begin{frame}{Football II}
  \begin{itemize}
  \item Fixtures lists
    \begin{itemize}
    \item No database right ({\it Fixtures Marketing} C-444/02)
    \item No independent investment in obtaining, verifying or presenting
    \item Big shock. What does it mean?
    \end{itemize}
  \item Football data
    \begin{itemize}
    \item Database right ({\it Football Dataco v Stan James} - Court of Appeal)
    \item Not tested in Europe
    \end{itemize}
  \end{itemize}
  
\end{frame}

\begin{frame}{Infringement}
Infringement tells you the extent of a right. For database right:
  \begin{itemize}
  \item Extraction (of a substantial part)
    \begin{itemize}
    \item i.e. {\bf copying}
    \item occurs if a substantial part of the data appears elsewhere
    \item can be done by hand, need not end up with same schema
    \end{itemize}
  \item Re-utilization (of a substantial part)
    \begin{itemize}
    \item i.e. making available to the public
    \item generally: publishing on the web
    \item meta search engine ({Innoweb v Wegener} C-202/12)
    \end{itemize}
  \item Repeated and systematic insubstantial extraction and/or re-utilisation
    \begin{itemize}
    \item conflicts with normal exploitation of database
    \item or unreasonably prejudice's maker's legitimate interests
    \end{itemize}
  \end{itemize}
\end{frame}

\begin{frame}{Database Copyright}
  \begin{itemize}
  \item An instance of copyright
  \item Author's ``own intellectual creation'' in:
    \begin{itemize}
    \item selection
    \item arrangement
    \end{itemize}
of the {\bf contents} of the database
  \item Not football fixtures lists ({Football Dataco v Yahoo!} C-604/10)
  \end{itemize}
\end{frame}

\begin{frame}{Example: German Poems}
  \begin{itemize}
  \item 100 most important poems in German literature between
1730 and 1900
\begin{itemize}
\item significant data mining effort given to
\item German Professor for final decision
\end{itemize}
\item Used as inspiration for another list
  \begin{itemize}
  \item infringement of database right by extraction ({Directmedia Publishing v
  Albert-Ludwigs-Universität Freiburg} C-304/07)
\item infringement of database copyright (Federal Court of Justice)
  \end{itemize}
  \end{itemize}
\end{frame}

\begin{frame}{Access Rights}
  \begin{itemize}
  \item Scraping
  \item Using robots
  \end{itemize}
\end{frame}

\begin{frame}{Unlawful Access}
  \begin{itemize}
  \item S.1 Computer Misuse Act 1990
    \begin{itemize}
    \item intent to secure access
    \item access is unauthorised
    \item defendant knows it is unauthorised
    \end{itemize}
  \item Applies to data in or defendant accessing from the UK
  \item Scraping not (yet) been prosecuted
  \end{itemize}
  
\end{frame}

\begin{frame}{Publishing}
Dead or alive?
\begin{itemize}
\item 100 greatest poems in German... [Dead]
\item OpenCorporates [Alive]
\end{itemize}
Options
\begin{itemize}
\item Dead (time insensitive) - licensing
\item Alive (time sensitive) - access control
\end{itemize}
  
\end{frame}

\begin{frame}{Licensing}
  \begin{itemize}
  \item Commercial terms
  \item Open
    \begin{itemize}
    \item attribution
    \item viral
    \item \sout{non-commercial}
    \end{itemize}
  \end{itemize}
\end{frame}

\begin{frame}{Open Licenses}
  \begin{itemize}
  \item Open Government Licence (not just the public sector)
  \item Creative Commons (all works)
    \begin{itemize}
    \item CC0
    \item CC-BY (attribution)
    \item CC-BY-SA (viral)
    \end{itemize}
  \item Open Data Commons (just databases)
    \begin{itemize}
    \item PDDL 
    \item ODC-BY
    \item ODbL
    \end{itemize}
  \end{itemize}
\end{frame}

\begin{frame}{Viral Open Licences}
  \begin{itemize}
  \item Creative Commons 4.0 Attribution Share-Alike (CC-BY-SA)
    \begin{itemize}
    \item one international licence
    \item now covers databases as well as their contents
    \end{itemize}
  \item Open Data Commons Open Database Licence (ODbL)
    \begin{itemize}
    \item purely for the database (not contents)
    \item also a contract (belt and braces)
    \item prevents derivative works
    \end{itemize}
  \end{itemize}
\end{frame}

\begin{frame}{OpenCorporates}
Example: OpenCorporates
\begin{itemize}
\item API access to corporate data
\item Free API
  \begin{itemize}
  \item data - ODbL
  \item must demonstrate contribution to the community
  \end{itemize}
\item Commercial API
  \begin{itemize}
  \item data - without share-alike restriction
  \item fee for use
  \end{itemize}
\end{itemize}
\end{frame}

\end{document}
